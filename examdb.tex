\documentclass[11pt]{exam}%
\usepackage{lastpage}%
\usepackage{amssymb, amsfonts, latexsym, verbatim, xspace, setspace,tikz,multicol,amsmath,multirow}%
\usepackage[margin=1in]{geometry}%
\usepackage{ragged2e}%
%
\usetikzlibrary{plotmarks}%
\singlespacing%
\parindent 0ex%
\newcommand{\class}{Math 9C: Calculus}%
\newcommand{\term}{2018W}%
\newcommand{\examnum}{Final}%
\newcommand{\examdate}{03/23/2018}%
\newcommand{\timelimit}{180min}%
%
\begin{document}%
\normalsize%
\pagestyle{head}%
\firstpageheader{}{}{}%
\runningheader{\class}{\examnum\ - Page \thepage\ of \numpages}%
\runningheadrule%
\begin{flushright}%
\begin{tabular}{p{2.8in} r l}%
\textbf{\class} & \textbf{Name (Print):} & \makebox[1.9in]{\hrulefill}\\
\textbf{\term} &&\\
\textbf{\examnum} &\textbf{Discussion TA:}&\makebox[1.9in]{\hrulefill}\\
\textbf{\examdate} &&\\
\textbf{Time Limit: \timelimit} &\textbf{Discussion time:}&\makebox[1.9in]{\hrulefill}%
\end{tabular}%
\\%
\end{flushright}%
\rule[1ex]{\textwidth}{.1pt}%
\newcommand{\boxwidth}{0.8cm}%

This exam contains \numpages\ pages (including this cover page) and \numquestions\ problems.  Check to see if any pages are missing.  Enter all requested information on the top of this page, and put your initials on the top of every page, in case the pages become separated.\\
\\
You may \textbf{NOT} use your books, notes, or any calculator on this exam.\\
\\
You are required to show your work on each problem on this exam.  The following rules apply:\\
\\
%
\begin{minipage}[t]{3.7in}%
\vspace{0pt}%
\begin{itemize}%
\item%

\textbf{Organize your work}, in a reasonably neat and coherent way, in the space provided. Work scattered all over the page without a clear ordering will receive very little credit.
%
\item%

\textbf{Mysterious or unsupported answers will not receive full credit}. A correct answer, unsupported by calculations, explanation, or algebraic work will receive no credit; an incorrect answer supported by substantially correct calculations and explanations might still receive partial credit.
%
\item%

Please cross (or erase) \textbf{everything} you don't want. If both the correct solution and the wrong solution appear simultaneously, we will just grade the wrong one.
%
\item%

If you need more space, use the back of the pages; clearly \textbf{indicate} when you have done this.
%
\item%

All problems should be answered in \textbf{exact values}, not decimal approximations (unless instructed explicitly to do so.%
\end{itemize}%
Do not write in the table to the right.%
\end{minipage}%
\hfill%
\begin{minipage}[t]{2.3in}%
\vspace{0pt}%
\gradetablestretch{2}%
\vqword{Problem}%
\addpoints%
\gradetable[v]%
\end{minipage}%
\newpage%
\addpoints%
\begin{questions}%
\question%

Evaluate
\[\int_0^2\int_0^3\int_{-\sqrt{9-x^2}}^{\sqrt{9-x^2}}\,dy\,dz\,dx.\]
%
\begin{solution}%

\[\begin{split}
    \int_0^2\int_0^3\int_{-\sqrt{9-x^2}}^{\sqrt{9-x^2}}\,dy\,dz\,dx=&\int_0^2\int_0^3y\rvert_{-\sqrt{9-x^2}}^{\sqrt{9-x^2}}\,dz\,dx=\int_0^2\int_0^32{\sqrt{9-x^2}}\,dz\,dx\\
    =&\int_0^2\left.2{\sqrt{9-x^2}}\cdot z\right\rvert_0^3\,dx=\int_0^22{\sqrt{9-x^2}}\cdot3\,dx\\
    =&\int_0^26\sqrt{9-x^2}\,dx=\left.3\left(x\sqrt{9-x^2}+9\arcsin(\frac{x}{3})\right)\right\rvert_0^2\\
    =&3\left(2\sqrt5+9\arcsin(\frac23)\right).
\end{split}\]
\begin{remark}
We can directly solve the integral using calculus techniques. However sometimes we need to recognize the region based on the formula. We take this as an example.

First we try to understand the region. It is: $-\sqrt{9-x^2}\leq y\leq\sqrt{9-x^2}$, $0\leq z\leq 3$ and $0\leq x\leq 2$. Since $z$ part has nothing to do with $x$ and $y$, we can first focus on $xy$-plane and see what it looks like. Here we have a curve $x^2+y^2=9$. For each $0\leq x\leq 2$, $-\sqrt{9-x^2}\leq y\leq\sqrt{9-x^2}$. Then we have the following region:
\[\begin{tikzpicture}
\begin{axis}[scale=0.8,
axis equal,
axis lines=middle,
xmin=-4, xmax=4, ymin=-4, ymax=4,xlabel=$x$, ylabel=$y$
]
\addplot[name path=top,samples=300,domain=-3:3] (x,{sqrt(9-x^2)});
\addplot[name path=but,samples=300,domain=-3:3] (x,{-sqrt(9-x^2)});
\addplot[fill=blue,opacity=0.4]fill between[of=top and but, soft clip={domain=0:2}];
\draw[dashed] (2,-2.5)--(2,2.5);
\end{axis}
\end{tikzpicture}\]
When considering $z$, since $z$ is not involving, we simply move the shaded region from $z=0$ to $z=3$. We get
\[
\begin{tikzpicture}
\begin{axis}[ticks=none,
scale=0.8,
view={120}{45},
axis lines=middle,
xmin=-4, xmax=4, ymin=-4, ymax=4, zmin=0,zmax=4,
xlabel=$x$, zlabel=$z$
]
\addplot[name path=top,samples=300,domain=-3:3] (x,{sqrt(9-x^2)});
\addplot[name path=but,samples=300,domain=-3:3] (x,{-sqrt(9-x^2)});
\draw[fill=blue,opacity=0.2] plot[smooth,samples=100,domain=0:2] (\x,-{sqrt(9-(\x)*(\x))}) --
    plot[smooth,samples=100,domain=2:0] (\x,{sqrt(9-(\x)*(\x))});
\draw node[above left]at (0,4,0){$y$};

\draw[dashed] (2,-2.5)--(2,2.5);
\end{axis}
\end{tikzpicture}\Rightarrow
\begin{tikzpicture}
\begin{axis}[scale=0.8,ticks=none,
view={120}{45},
axis lines=middle,
xmin=-4, xmax=4, ymin=-4, ymax=4, zmin=0,zmax=4,
xlabel=$x$,
]
\addplot[name path=top,samples=300,domain=-3:3] (x,{sqrt(9-x^2)});
\addplot[name path=but,samples=300,domain=-3:3] (x,{-sqrt(9-x^2)});
\draw[fill=blue,opacity=0.2] plot[smooth,samples=100,domain=0:2] (\x,-{sqrt(9-(\x)*(\x))}) --
    plot[smooth,samples=100,domain=2:0] (\x,{sqrt(9-(\x)*(\x))});
\draw[dashed] (2,-2.5)--(2,2.5);

\draw node[left]at (0,0,4){$z$};
\draw node[above left]at (0,4,0){$y$};

\addplot3[samples y=1,domain=-3:3,dashed] (x,{sqrt(9-x^2)},3);
\addplot3[name path=but,samples y=1,domain=-3:3,dashed] (x,{-sqrt(9-x^2)},3);
\draw[fill=blue,opacity=0.2] plot[smooth,samples=100,domain=0:2] (\x,-{sqrt(9-(\x)*(\x))},3) --
    plot[smooth,samples=100,domain=2:0] (\x,{sqrt(9-(\x)*(\x))},3);
\draw[dashed] (2,-2.5,3)--(2,2.5,3);
\draw[dashed] (0,-3,3)--(0,3,3);
\draw[dashed] (0,-3,0)--(0,-3,3);
\draw[dashed] (0,3,0)--(0,3,3);
\draw[dashed] (2,{-sqrt(5)},0)--(2,{-sqrt(5)},3);
\draw[dashed] (2,{sqrt(5)},0)--(2,{sqrt(5)},3);
\end{axis}
\end{tikzpicture}\]
\end{remark}
%
\end{solution}%
\end{questions}%
\end{document}